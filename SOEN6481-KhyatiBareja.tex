
% This is samplepaper.tex, a sample chapter demonstrating the
% LLNCS macro package for Springer Computer Science proceedings;
% Version 2.20 of 2017/10/04
%
\documentclass[runningheads]{llncs}
\usepackage{amsmath}
\usepackage{hyperref}
%
\usepackage{graphicx}
% Used for displaying a sample figure. If possible, figure files should
% be included in EPS format.
%
% If you use the hyperref package, please uncomment the following line
% to display URLs in blue roman font according to Springer's eBook style:
% \renewcommand\UrlFont{\color{blue}\rmfamily}

\begin{document}
%
\title{What should I consider when setting
up software tools I will be using to
coordinate many interrelated projects?}
%
%\titlerunning{Abbreviated paper title}
% If the paper title is too long for the running head, you can set
% an abbreviated paper title here
%
\author{Khyati Bareja | Student ID : 40221300}
%

%
\institute{Computer Science and Software Engineering , Concordia University \\ 
\vspace{10pt}  VCS: Github: \href{https://github.com/Khyati-Bareja/SOEN6481-TAS}{https://github.com/Khyati-Bareja/SOEN6481-TAS}}

%
%
\maketitle

\tableofcontents

\newpage

\section{Abstract}
The approaches and crucial factors that must be taken into account when creating software tools to manage several connected projects are highlighted in this study. The emphasis is on how important it is to comply with industry norms, organizational requirements, regulatory frameworks, and program management office suggestions, and also using a strong Project or Program Management Information System, to exercise control and make work collaboration easier. The adoption of software packages with sophisticated capabilities is highlighted, as is the strategic way of organizing online data for easy accessibility by remote project teams. Additionally, in order to improve organizational clarity, the study supports, careful recording of the staffing hierarchy inside the program and assigning tasks to a designated program owner. It is advised that centralized, sophisticated technologies be used to support general uniformity. One of the most important practices for attaining complete and unified project management is having program staff participate in project starting workshops and planning meetings.

\section{Introduction}

\subsection{Motivation}
The corporate landscape of today is multifaceted. Managers must be able to act quickly, distribute limited resources effectively, and maintain focus. Management faces a variety of difficulties with businesses that are working on several projects at once which are interrelated. Particular issues arise for project managers in charge of several projects with various scopes, levels of complexity, and deadlines. These may have to do with throughput times and conflicts over resources. When limited resources are not properly balanced, the organization is frequently under more strain, which results in poor information quality and longer project lead times.[1]

Hence, setting up an effective project or program management information system(PMIS) and utilizing software tools to coordinate interrelated projects becomes necessary for maintaining coordination and control over ongoing activities and to organize online information so that members of remote project teams may quickly locate the information they require. This may in turn help the management to coordinate their distributed and interrelated projects and contribute in better decision making, by avoiding resource overlapping,provisioning the easy delegation of work, evaluation and estimation of costs and plan changes accordingly and in parallel.[1]
\subsection{Problem Statement}

The problem statement addressed here is about the factors to consider when setting up software tools to coordinate multiple interrelated projects. This includes:
\begin{enumerate}
  \item Assessing all the information requirements that the program will need to comply with, determining how to meet them, and establishing an effective project or program management information system (PMIS) to control and coordinate ongoing work and coordinating program plans.

 \item Monitoring Progress via use of common computer scheduling tools for all projects within the program.

\end{enumerate}


\subsection{Objectives}
\begin{enumerate}

\item Evaluate program information requirements for compliance with rules and guidelines and establish or leverage infrastructure for storing project data, considering long-term obligations.
\item Setting up up an effective Program Management Information System (PMIS) for control and coordination.
\item Explore advanced options to enhance information infrastructure capabilities.
\item Clearly documenting staffing hierarchy, roles, responsibilities, and contact information to better coordinate among inter-related projects
\item Ensuring access to common scheduling tools for all projects.
\item Adopt compatible project management software, providing training and expertise to coordinate program plans.
\item Standardizing processes for status collection and reporting, ensuring consistency throughout the programs.

\end{enumerate}

\section{Background Material}

\section{Methods \& Methodology}

\section{Results Obtained}

\section{Conclusions and Future Works}

\subsection{Limitations to Solution}

\subsection{Suggested Improvements}

\subsection{Applications in Real World}

\subsection{Conclusion}


\section{References}
\begin{enumerate}
  \item [1]M. C. J. Caniëls and R. J. J. M. Bakens, “The effects of Project Management Information Systems on decision making in a multi project environment,” International Journal of Project Management, vol. 30, no. 2, pp. 162–175, Feb. 2012, doi: https://doi.org/10.1016/j.ijproman.2011.05.005.
‌
‌

\end{enumerate}


\section{Acknowledgements}
I am grateful to ChatGPT, Perplexity, and a few YouTube channels for their important help and insights, which made this effort much better. Furthermore, references from academic journals that can be accessed via Google Scholar have been included.



\end{document}
