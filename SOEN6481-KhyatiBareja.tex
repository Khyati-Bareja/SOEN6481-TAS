
% This is samplepaper.tex, a sample chapter demonstrating the
% LLNCS macro package for Springer Computer Science proceedings;
% Version 2.20 of 2017/10/04
%
\documentclass{llncs}
\usepackage{graphicx}
\makeatletter
\@twosidefalse
\@mparswitchfalse
\makeatother
\usepackage{amsmath}
\usepackage{hyperref}
\usepackage{setspace}
\onehalfspacing
\usepackage{float}
\pagestyle{headings}
\setcounter{tocdepth}{8}
\hypersetup{%
    pdfborder = {0 0 0}
}
\renewenvironment{abstract}{
  \begin{center}%
    {\large\bfseries Abstract}
  \end{center}
  \quotation
}
{
  \endquotation
}
\usepackage{sectsty}
\sectionfont{\fontsize{15}{14}\selectfont}
\subsectionfont{\fontsize{13}{13}\selectfont}

\usepackage{graphicx}
% Used for displaying a sample figure. If possible, figure files should
% be included in EPS format.
%
% If you use the hyperref package, please uncomment the following line
% to display URLs in blue roman font according to Springer's eBook style:
% \renewcommand\UrlFont{\color{blue}\rmfamily}
\begin{document}
%

\title{

\begin{figure}[ht]
    \centering
    \includegraphics[width=1\linewidth]{concordia.jpeg}
    \label{fig:example}
\end{figure}

What should I consider when setting
up software tools I will be using to
coordinate many interrelated projects?\vspace{5pt}}
%
%\titlerunning{Abbreviated paper title}
% If the paper title is too long for the running head, you can set
% an abbreviated paper title here
%
\author{\large \textbf{Khyati Bareja - Student ID: 40221300}\vspace{20pt}}
%
%
\institute{\large \textbf{SOEN 6841- Software Project Management} \\
\vspace{15pt} \textbf{Advisor: Pankaj Kamthan} \\
\vspace{60pt} \textbf{Computer Science and Software Engineering , Concordia University} \\
\vspace{15pt} \textbf{ VCS: Github: \href{https://github.com/Khyati-Bareja/SOEN6481-TAS}{https://github.com/Khyati-Bareja/SOEN6481-TAS}}}

{\def\addcontentsline#1#2#3{}\maketitle}
\newpage

\section{Abstract}
The approaches and crucial factors that must be taken into account when creating software tools to manage several connected projects are highlighted in this study. The emphasis is on how important it is to comply with industry norms, organizational requirements, regulatory frameworks, and program management office suggestions, and also using a strong Project or Program Management Information System, to exercise control and make work collaboration easier. The adoption of software packages with sophisticated capabilities is highlighted, as is the strategic way of organizing online data for easy accessibility by remote project teams. Additionally, in order to improve organizational clarity, the study supports, careful recording of the staffing hierarchy inside the program and assigning tasks to a designated program owner. It is advised that centralized, sophisticated technologies be used to support general uniformity. One of the most important practices for attaining complete and unified project management is having program staff participate in project starting workshops and planning meetings.



\tableofcontents
\newpage

\section{Introduction}

\subsection{Motivation}
The corporate landscape of today is multifaceted. Managers must be able to act quickly, distribute limited resources effectively, and maintain focus. Management faces a variety of difficulties with businesses that are working on several projects at once which are interrelated. Particular issues arise for project managers in charge of several projects with various scopes, levels of complexity, and deadlines. These may have to do with throughput times and conflicts over resources. When limited resources are not properly balanced, the organization is frequently under more strain, which results in poor information quality and longer project lead times.~\cite{refpaper1}

Hence, setting up an effective project or program management information system(PMIS) and utilizing software tools to coordinate interrelated projects becomes necessary for maintaining coordination and control over ongoing activities and to organize online information so that members of remote project teams may quickly locate the information they require. This may in turn help the management to coordinate their distributed and interrelated projects and contribute in better decision making, by avoiding resource overlapping,provisioning the easy delegation of work, evaluation and estimation of costs and plan changes accordingly and in parallel.~\cite{refpaper1}

\subsection{Problem Statement}
The problem statement addressed here is about the factors to consider when setting up software tools to coordinate multiple interrelated projects. This includes:
\begin{enumerate}
  \item Assessing all the information requirements that the program will need to comply with, determining how to meet them, and establishing an effective project or program management information system (PMIS) to control and coordinate ongoing work and coordinating program plans.

 \item Monitoring Progress via use of common computer scheduling tools for all projects within the program.

\end{enumerate}


\subsection{Objectives}
Objectives Depends on majorly three factors,
\begin{enumerate}
\item \textbf{Organizational requirements}:
\subitem Evaluate program information requirements for compliance with rules and guidelines and establish or leverage infrastructure for storing project data, considering long-term obligations.
\item \textbf{Program management office (PMO) recommendations}: \subitem Setting up up an effective Program Management Information System (PMIS) for control and coordination.
\subitem Explore advanced options to enhance information infrastructure capabilities.
\item \textbf{Regulations and industry standards and Access}:
\subitem Clearly documenting staffing hierarchy, roles, responsibilities, and contact information to better coordinate among inter-related projects, as per industry standards.
\subitem Ensuring access to common scheduling tools for all projects.
\subitem Adopt compatible project management software, providing training and expertise to coordinate program plans.
\subitem Standardizing processes for status collection and reporting, ensuring consistency throughout the programs.

\end{enumerate}

\section{Background Material}
The topic of Interdependecies and importance of mitigating its impact on the project has been in research from some time now. On the basis of proven studies there are various categorization of these interdependencies and how they affect the project management. \\
\textbf{Types of Interdependencies between Multiple Projects:}
\subitem \textbf{External Interdependencies:} These are influenced by factors external to the organization, such as social and economic changes. For example, a sudden change in market conditions can lead to priority variations in the projects within a portfolio or outside.~\cite{refpaper2}
\subitem \textbf{Internal Interdependencies:} These arise when the resource requirements or benefits of one project are significantly affected by the selection or rejection decisions of other projects within the set. For instance, an unexpected delay in one project can impact other dependent projects, leading to an overall delay in the completion time of a new product or service.~\cite{refpaper2}
\subitem \textbf{Resource Interdependencies:} These occur when there is a need to share resources or wait for scarce resources until another project releases them. Improper management of resource interdependencies can result in resource waste.~\cite{refpaper2}
\subitem\textbf{Technology Interdependencies:} These involve leveraging common technology across multiple projects. If not properly managed, technology interdependencies can lead to schedule slippage.~\cite{refpaper2}
\subitem \textbf{Technical Interdependencies:} These arise from the need to address technical challenges that cut across multiple projects. Ineffective management of technical interdependencies can result in budget waste. \subitem\textbf{Learning-based Interdependencies:} These stem from the need to incorporate the capabilities and knowledge gained from previous projects. Poor management of learning-based interdependencies can hinder knowledge diffusion and innovation.~\cite{refpaper2} \\
Along with the above mentioned there exist other landscape of interdependencies among project activities and tasks as well,that can further be categorized on varying level of interdependence as  'flow,' 'fit,' and 'sharing.'~\cite{refpaper5} \\
\textbf{Problems Caused by Poor Project Interdependencies Management can be:}
\subitem \textbf{Wastage of Resources:} Inadequate consideration of project interdependencies can result in inefficient resource utilization, leading to resource waste.~\cite{refpaper2}~\cite{refpaper3}
\subitem \textbf{Schedule Slippage:} When interdependent projects are not properly managed, a delay in one project can propagate to other interconnected projects, causing overall schedule slippage.~\cite{refpaper2}~\cite{refpaper3}
\subitem \textbf{Budget Wastage:} Failure to consider interdependencies among projects during the selection process can lead to unnecessary duplication of efforts and resources, resulting in budget waste.~\cite{refpaper2}
\subitem\textbf{Inter-project Competition:} When projects start competing for scarce resources, it can lead to conflicts and hinder the achievement of project objectives.~\cite{refpaper2}~\cite{refpaper4}
\subitem\textbf{Impact on Cost:} The cost of managing project interdependencies can be impacted in various ways. Poor management of interdependencies can lead to resource waste, schedule slippage, and budget waste, all of which can increase project costs. On the other hand, effective management of interdependencies can optimize resource utilization, prevent delays, and reduce unnecessary expenses, resulting in cost savings.~\cite{refpaper2}

Hence, enhancing project management by equipping project managers with sophisticated analytical tools, will envision to help manage the complexity of interdependent projects by offering insights into information flow and critical activities for more informed decision-making and optimized project outcomes.~\cite{refpaper5}


\section{Methods \& Methodology}
%Managing knowledge

%----Tools to be used --- to use citation [2] here
\section{Results Obtained}

\section{Conclusions and Future Works}

\subsection{Limitations to Solution}

\subsection{Suggested Improvements}

\subsection{Applications in Real World}

\subsection{Conclusion}



\begin{thebibliography}{5}
\bibitem{refpaper1}
  M. C. J. Caniëls and R. J. J. M. Bakens, "The effects of Project Management Information Systems on decision making in a multi-project environment," \textit{International Journal of Project Management}, vol. 30, no. 2, pp. 162–175, Feb. 2012, doi: \url{https://doi.org/10.1016/j.ijproman.2011.05.005}.
\bibitem{refpaper2}
  S. Bathallath, Å. Smedberg, and H. Kjellin, "Managing project interdependencies in IT/IS project portfolios: a review of managerial issues," \textit{International Journal of Information Systems and Project Management}, vol. 4, no. 1, pp. 67–82, 2016, doi: \url{https://doi.org/10.12821/ijispm040104}.
‌\bibitem{refpaper3}
  R. Sweetman and K. Conboy, "Portfolios of Agile Projects," \textit{Project Management Journal}, vol. 49, no. 6, pp. 18–38, Oct. 2018, doi: \url{https://doi.org/10.1177/8756972818802712}.
\bibitem {refpaper4}
 M. Engwall and A. Jerbrant, "The resource allocation syndrome: the prime challenge of multi-project management?," \textit{International Journal of Project Management}, vol. 21, no. 6, pp. 403–409, Aug. 2003, doi: \url{https://doi.org/10.1016/s0263-7863(02)00113-8}.
\bibitem{refpaper5}
H. Bashir, U. Ojiako, A. Marshall, M. Chipulu, and A. A. Yousif,
\textit{The analysis of information flow interdependencies within projects},
\textit{Production Planning \& Control}, pp. 1–17, Sep. 2020,doi: \url{https://doi.org/10.1080/09537287.2020.1821115}.
‌
\end{thebibliography}



\section{Acknowledgements}
I am grateful to ChatGPT, Perplexity, and a few YouTube channels for their important help and insights, which made this effort much better. Furthermore, references from academic journals that can be accessed via Google Scholar have been included.




\end{document}